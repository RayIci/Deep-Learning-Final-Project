\section*{Paper Qualitative Results}

\subsection*{Comparison of GAN Variants}

We compare the following GAN architectures:  
\begin{itemize}
    \item \textbf{GAN Baseline:} The basic GAN model without any specific improvements for text-image matching.  
    \item \textbf{GAN-CLS:} A GAN model incorporating an image-text matching discriminator .  
    \item \textbf{GAN-INT:} A GAN variant utilizing text manifold interpolation
    \item \textbf{GAN-INT-CLS:} A model combining both text manifold interpolation and the image-text matching discriminator.  
\end{itemize}

\subsection*{Results on the CUB Dataset}
Qualitative results for the CUB dataset:  
\begin{itemize}
    \item The \textbf{GAN Baseline} and \textbf{GAN-CLS} models correctly reproduce some color information. However, the generated images do not appear realistic.  
    \item \textbf{GAN-INT} and \textbf{GAN-INT-CLS} models produce plausible bird images that match either all or part of the captions.  
    \item Additional robustness analysis for each GAN variant on the CUB dataset is provided in the supplement.  
\end{itemize}

\subsection*{Results on the Oxford-102 Dataset}
Qualitative results for the Oxford-102 Flowers dataset are presented in Figure 4:  
\begin{itemize}
    \item All four models are capable of generating plausible flower images that align with their respective captions.  
    \item The \textbf{GAN Baseline} exhibits the highest variety in flower morphology, especially when the caption does not specify petal types.  
    \item Other models, such as \textbf{GAN-CLS}, \textbf{GAN-INT}, and \textbf{GAN-INT-CLS}, generate more class-consistent flower images.  
    \item It is speculated that generating flowers is easier than birds 
    due to structural regularities in bird species, making it simpler 
    for the discriminator to identify fake birds compared to fake flowers.  
\end{itemize}

\subsection*{Additional Results}
Supplementary materials include additional examples for the following:  
\begin{itemize}
    \item \textbf{GAN-INT} and \textbf{GAN-INT-CLS} models on both CUB and Oxford-102 datasets.  
    \item \textbf{Vanilla GAN:} An end-to-end variant of GAN-INT-CLS that does not rely on pre-training the text encoder \(\phi(t)\).  
\end{itemize}
