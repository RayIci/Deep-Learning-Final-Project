\section*{Summary}

%Il Paper che si è deciso di analizzare è un documento risalente al 
%2016 che mostra lo sforzo di 6 ricercatori provenienti dall'Università 
%del Michigan e di Saarbrucken (Germania) . 
%Il documento mostra come l'utilizzo delle GAN (Generative Adversial Newtworks)
%permette un avanzamento nella generazione di immagini sintetiche partendo 
%da una descrizione testuale . 
%Esso mette a confronto il metodo proposto dalla ricerca e le 
%precedenti architetture che per quanto lontane dal traguardo descritto 
%sono in grado di ottenere rappresentazioni di feature testuali valide . 
%In particolare il paper vuole mostrare l'efficaccia del modello nel 
%generare immagini di uccelli e fiori partendo da una precisa descrizione 
%testuale degli stessi .
The paper that has been chosen for analysis is a document from 2016 
that showcases the effort of six researchers from the University of 
Michigan and Saarbrücken (Germany). 
The document shows how the use of GANs (Generative Adversarial Networks) 
allows for advancements in the generation of synthetic images starting 
from a textual description. 
It compares the method proposed by the research with previous architectures
that, although far from the described goal, are capable of obtaining 
valid textual feature representations. 
In particular, the paper aims to demonstrate the effectiveness of 
the model in generating images of birds and flowers based on a precise 
textual description of them.