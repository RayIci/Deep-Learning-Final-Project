\section*{ Methodologies }
All the methods are based on a GAN architecture with a Generator G 
and a Discriminator D . \\ 
And use ($\varphi$) as Text Encoder that is a character-level convolutional-
recurrent neural network .

The generator is defined as :
\[
G : \mathbb{R}^Z \times \mathbb{R}^{T} \rightarrow \mathbb{R}^D,
\]
And the discriminator is defined as :  
\[
D : \mathbb{R}^D \times \mathbb{R}^{T} \rightarrow \{0, 1\},
\]
where 
\begin{itemize}
    \item ${D}$: is the dimension of the generated image
    \item ${Z}$: is the dimension of the input noise 
    \item ${T}$: is the dimension of the text emdedded with $\varphi$
\end{itemize}

\subsection*{- Generator Part -}
First of all the generated image is defined as :
\[
\hat{x} \leftarrow G(z, \varphi(t))
\]
This Generator take as input a noise vector of dimension Z 
(In our implementation Z = 100) whose values are sampled from a Gaussian 
distribution between 0 and 1
\\
\[
z \in \mathbb{R}^Z \sim \mathcal{N}(0, 1)
\]
and the vector coming from the Text Encoder $\varphi$ for the specific 
text description t obtaining $\varphi(t)$ .
\\
First part - Projection \\
The first step of the Generator Net is to reduce the dimension of 
$\varphi(t)$ using one fully-connected layer to project 
to the dimension of the projection P . \\
This part continue using a Normalization Layer (BatchNorm1d)
and a Leaky-Relu as activation function with a specific negative slope .  
So the values on the vector can be negative and positive depending on the 
situation .

In our implementation $\varphi$ correspond to CLIP while \\
T is equal to 512 and \\
P is equal to 128 \\
After this first part an 128-dim vector is obtained and defined as :
\[
 p = projection( \varphi(t) )
\]

Second part - Concatanation \\
To compose the Latent Space vector (LSV) the paper suggested to concatenate 
the noise vector z with the projected vector p .

In our specific implementation the latent space has a dimension of H (H=228)
which is simple the concatation of the two vector of dimension T=100 
and P=128 .

\[
h = concat( z , p )
\]

Third part - Final Net to generated the image 64x64 RGB 
As input of the Final Net part of the Generator is the Latent Space Vector 
(LSV) of dimension H (H=228). 

This Final Net is composed of 4 set of layers in which the first one is a 
Convolutional Transpose 2D Layers ( "to up-sample" ) followed by a Normalization layer 
and Relu as activation function so we only have values greater than 0 .

In our implementation the initial dimension of the vector was (228,1,1) but 
after only the first Convolutional Transpose layer it is projected into an higher dimension of 
(512,8,8) .

After using the other 3 set of layers the dimension will be equal to (64,32,32) .
The last layer is useful because permit to obtain an image of the shape 
(3,64,64) and is composed of a Convolutional Transpose 2D Layer and Tanh as activation function .

Using Tanh with in input values coming from the last layer that can be positive and 
negative as well due to the weight of the 2D Convolutional Transpose Layer we will obtain an 
image 64x64 with 3 channel RGB with values that ranges from -1 to 1 due to Tanh .
\[
\hat{x} = generate( h )
\]
And G is defined as this 3 Parts applied in order one after another  . 


\subsection*{- Discriminator Part -}
The Discriminator is defined as this :
\[
D ( \hat{x} , \varphi(t) ) = \left\{ x_i \in [0, 1] \mid i = 0, 1, \dots, 120 \right\}
\]
It takes in input the generated image $\hat{x}$ and depending
on the situation an encoded version of the text description $\varphi(t)$
related or not to image generated .

Similarly to what we have seen in the Generator Part the Discriminator is 
sub-divided in 3 Parts or Networks . 

First Net - Down-Sampling 
The First Part is composed of a several (in our implementation 4) set 
of Convolutional 2D Layers ("to down-sample") 
followed by a normalization Layer (BatchNorm2d except the first set)
and Leaky-Relu as activation function . 
After passing through this Net the generated image of a shape of (3,224,244)
will have a dimension of (512,14,14) .
\[
 q = DownSampling( \hat{x} )
\]

Second Part - Projection Text Emdedding and Concatenation

This part consist of two phases and it take as argument the down-sampled generated 
image ${q}$ and the encoded version of the text description $\varphi(t)$ .
The first phase consist in projecting $\varphi(t)$
in a lower-dimension P (P=128) using one fully-connected layer and normalization layer 
followed by a Leaky-Relu as activation function in the exact same way that we have done in 
the Generator .
After this first phase a 128-dim vector is obtained and defined as :
\[
 p = projection( \varphi(t) )
\]
The last task of the first phase was to adapt the project dimension of p 
( that is (128,1,1) ) in such a way to be compatibile with the second phase 
which concatenate p with q . 
So the dimension of p is squeezed into a dimension that allow to concatenate
the two data and it became (128,14,14)
\[
 \hat{p} = squeeze( p )
\]

The second Phase consist of the concatenation of $q$ and $\hat{p}$ 
which result in a dimension of (512+128, 14, 14) = (256, 640, 14, 14)
This concatation permit to obtain the Latent vector c :
\[
 c = concatenate( q , \hat{p} )
\]


Third Part - Final Net 
This Final Net it's only composed of One Convolutional 2D Layer and 
Sigmoid as activation function which map the input in a range that goes 
from 0 to 1 .
The first Convolution Layer modify the shape from (256, 640, 14, 14) 
to (512,4,4) while the sigmoid function change its shape into
( 1,11,11 ) .
At the end of this phase we flatten the 11x11 values into a vector of 
121-dimension .
So at the end for each generated image and text Emdedding we obtain a 
121-dim vector rappresenting the discriminator factors .
\[
 d\_loss = FinalNet( c )
\]

--------------------------
