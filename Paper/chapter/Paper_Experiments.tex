\section*{Experiment conducted by the Paper}

\subsection*{Datasets}

\subsubsection*{CUB Dataset}
The CUB dataset contains 11,788 bird images belonging to 200 distinct categories. 

\subsubsection*{Oxford-102 Dataset}
The Oxford-102 dataset consists of 8,189 flower images, grouped into 102 categories.

\subsubsection*{Data Splits}
For training and evaluation, datasets are split into class-disjoint subsets:  
\begin{itemize}
    \item \textbf{CUB:} 150 categories are used for training and validation, while the remaining 50 are used for testing.
    \item \textbf{Oxford-102:} 82 categories are assigned to training and validation, and 20 categories are reserved for testing.
\end{itemize}

\subsubsection*{Captions}
Each image in both datasets is accompanied by 5 captions. During training, a random image view (e.g., crop, flip) and one randomly selected caption are used for each mini-batch.

\subsection*{Text Encoder}

\subsubsection*{Encoder Architecture}
The text encoder employs a deep convolutional-recurrent network, combining a character-level ConvNet with a recurrent neural network (char-CNN-RNN). This architecture generates 1,024-dimensional embeddings for textual descriptions.

\subsubsection*{Text Embedding}
Text captions are embedded into a 1,024-dimensional space via structured joint embedding with GoogLeNet features. 

\subsubsection*{Pre-training}
Pre-training the text encoder accelerates the training of the generator and discriminator, enabling faster experimentation. While pre-training is not a strict requirement, end-to-end training results are provided in the supplement for completeness.

\subsubsection*{Generalization}
Qualitative results from the MS COCO validation set illustrate the approach's ability to generalize beyond the datasets used in training.

\subsection*{GAN Architecture}

\subsubsection*{Training Image Size}
All training images are resized to \(64 \times 64 \times 3\).

\subsubsection*{Text Feature Projection}
Text embeddings are projected to a 128-dimensional space before being concatenated with convolutional feature maps in both the generator and discriminator networks.

\subsubsection*{Training Process}
The generator and discriminator in the GAN-CLS architecture are updated 
in alternating steps to optimize the GAN architecture.

\subsubsection*{Hyperparameters}
The following hyperparameters are used during training:
\begin{itemize}
    \item Base learning rate: 0.0002.
    \item Optimizer: ADAM, with a momentum parameter of 0.5.
    \item Generator noise: Sampled from a 100-dimensional unit normal distribution.
    \item Mini-batch size: 64.
    \item Number of epochs: 600.
\end{itemize}

\subsubsection*{Implementation}
The implementation of the model is based on the \texttt{dcgan.torch2} 
framework.
