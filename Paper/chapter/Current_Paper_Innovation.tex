
\section*{Current Paper innovation ? }
Our technique differs from previous conditional GANs by focusing on text descriptions rather than class labels. 
This is the first design capable of differentiating from character to pixel level. 
The paper develop a manifold interpolation regularizer for the GAN generator, 
which enhances sample quality, including "zero-shot" categories on CUB.
It describes a model that generates 64x64 visually convincing pictures from text using a GAN. 
It differs from other models that just employ GANs for post-processing.
In practice it used a character-level text encoder and class-conditional GAN and 
focus in implementing a new architecture and using it on fine-grained  
image datasets described before ( CUB and Oxford Flowers ) .
Testing on MOCO dataset and test set disjoint from Training set 
can return a strong indicator on the performance of the system .
