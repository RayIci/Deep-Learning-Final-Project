\section*{Summary}
The first observation to make is that a longer version of this short report 
has been drafted, which contains more details on the implementation and 
theoretical aspects of the methods in the paper and also on our 
implementation. \\
The paper that has been chosen to be analysed is a 2016 Paper  
that showcases the effort of six researchers from the University of 
Michigan and Saarbrücken (Germany). 
The document shows how the use of GANs (Generative Adversarial Networks) 
allows for advancements in the generation of synthetic images starting 
from a textual description. 
It compares the method proposed by the research with previous architectures
that, although far from the described goal, are capable of obtaining 
valid textual feature representations. 
In particular, the paper aims to demonstrate the effectiveness of 
the model in generating images of birds and flowers based on a precise 
textual description of them.
In 2016, the ability of an AI system to generate realistic and 
coherent images from textual descriptions 
(such as "small red bird with a blue beak") was a current issue and 
far from being achieved.
It should be noted that it's necessary to use natural language 
and domain-specific attributes to describe the image to be generated.
\\
The difficulty of translating words into images may be 
divided into two subproblems.
First, learn a feature vector from a specific text 
based on the visualization we want to obtain .
Given these features through the use of a certain architecture, 
create a realistic and coherent image.
This paper describes a GAN that creates 64x64 images from text descriptions, 
utilizing a character-level encoder and a unique manifold interpolation regularizer 
for fine-grained datasets such as CUB and Oxford Flowers.
\\
During our work, we attempted to reimplement the version proposed by the Paper in order 
to determine its correctness and to be able to reapply it to different domains and tasks, 
but unfortunately, it was not possible for the reasons described in the "Experiments" section, 
while we successfully managed to implement a "toy" version with a different architecture.