\section*{ Methodologies }
[ADD HERE DESCRIPTION OF THE SECTIONS Scrivi breve introduzione su come è 
strutturata la parte di metodologia (esempio: In questa sezione trovate divisi per sezioni dataset, 
gan, ..., etc... usati nel paper con qualche commento su come e cosa abbiamo cambiato nella nostra implementazione )]

All the methods are based on a GAN architecture with a Generator G 
and a Discriminator D . \\ 
And use ($\varphi$) as Text Encoder that is a character-level convolutional-
recurrent neural network .

The generator is defined as :
\[
G : \mathbb{R}^Z \times \mathbb{R}^{T} \rightarrow \mathbb{R}^D,
\]
And the discriminator is defined as :  
\[
D : \mathbb{R}^D \times \mathbb{R}^{T} \rightarrow \{0, 1\},
\]
where 
\begin{description}
    \item[${D}$] is the dimension of the generated image
    \item[${Z}$] is the dimension of the input noise
    \item[${T}$] is the dimension of the text embedded with $\varphi$
\end{description}


\subsubsection*{Optimization Formula}
The optimization objective is defined as:  
\begin{equation}
    \min_{G} \max_{D} V(D, G) = \mathbb{E}_{x \sim p_{\text{data}}(x)}[\log D(x)] 
\end{equation}
\[
+ \mathbb{E}_{z \sim p_z(z)}[\log(1 - D(G(z)))]
\]
\begin{description}
    \item[$D(x)$] The Discriminator uses a Sigmoid activation function, mapping outputs to $[0, 1]$.
    \item[$G(x)$] The Generator outputs data in $\mathbb{R}^D$.
\end{description}


The Discriminator is maximized to distinguish real from fake data, 
while the Generator is minimized to make $G(z)$ indistinguishable from real data.  
Maximizing $D$ involves maximizing $\log D(x)$ for real data and $\log(1 - D(G(z)))$ for generated data.  
Alternatively, $G$ can also be optimized using $\log D(G(z))$ instead of $\log(1 - D(G(z)))$ to achieve the same result.


\subsection*{Paper Datasets}
All datasets share the following features: each image is paired with 5 text captions, 
and during training, a random caption and an augmented version of the image (e.g., crop, flip) 
are used for each mini-batch.

\begin{description}
    \item[\textbf{Caltech-UCSD Birds Dataset (CUB):}] 
    Contains 11,788 images of birds across 200 categories. The dataset is split into 150 categories 
    for training and validation, and 50 categories for testing.
    
    \item[\textbf{Oxford-102 Flowers Dataset:}] Includes 8,189 images of flowers grouped into 102 categories. 
    The dataset is split into 82 categories for training and validation, and 20 categories for testing.
    
    \item[\textbf{MS COCO Dataset:}] Used for testing with a broader range of general images and text descriptions.
\end{description}


\subsection*{Text and Image Encoder}

To encode textual descriptions and images, the original paper utilized a symmetric 
architecture combining text and image representations, following the approach of Reed et al. (2006). 
The dataset consists of $N$ samples, where each sample includes an image, 
its text description, and its class label. 

\paragraph{Encoders Overview}
The image encoder is based on GoogLeNet, producing 1,024-dimensional embeddings, 
while the text encoder employs a deep convolutional-recurrent architecture (char-CNN-RNN), 
also generating 1,024-dimensional text embeddings. 
These embeddings are mapped to a shared space, enabling 
compatibility between textual and visual modalities.

\paragraph{Training Objectives}
The main goal is to ensure that text descriptions are most compatible with images from
their respective classes. To achieve this, we compute the similarity between the 
encoded version of the input image $v$ and the current text description $t$ 
related to the current class $y$. The encoded versions, $\phi(v)$ and $\varphi(t)$, return vectors, 
and the similarity is computed via their dot product, which returns a scalar. 
This scalar represents how well the image and text description match. The goal is 
to maximize the correlation between the image and the text of the same class, 
ensuring that the image-text pairs that belong to the same class have the highest similarity.

\paragraph{Loss Function}
To optimize the encoders, a hinge loss or 0-1 loss function is used. 
This loss function penalizes the similarity score between mismatched image-text pairs, 
while encouraging high similarity scores for matching pairs. Specifically, 
the loss function encourages the model to predict the correct class $y$ by minimizing the hinge loss, defined as:
\[
\Delta( y, f(x) ) = \max (0 , 1 - y * f(x) )
\]

where $f(x)$ is the classifier output for the given image or text input. 
This results in a classification loss that forces the encoder to learn meaningful representations.

\paragraph{Our Implementation Details}
In our implementation, instead of using the original encoder described by Reed et al. (2006), 
we used CLIP (Contrastive Language-Image Pre-Training), a model developed by OpenAI. 
The CLIP model is pre-trained on large datasets and fine-tuned on the CUB and Oxford-102 
datasets to generate relevant vector representations for both images and text descriptions. 
Pretraining on large datasets accelerates the training process and improves the model's 
ability to generalize across different domains. Fine-tuning on domain-specific datasets such 
as CUB and Oxford-102 enhances the quality of the embeddings for the task at hand.


\subsection*{Paper's Architectures and our}
The GAN architecture used in the paper follows the implementation where all training images are resized 
to $64 \times 64 \times 3$. 
Text embeddings are projected into a 128-dimensional space and concatenated with convolutional feature 
maps in both the generator and discriminator networks. 
During training, the generator and discriminator are updated in alternating steps to optimize the model. 
The hyperparameters used include a base learning rate of 0.0002, ADAM optimizer with a momentum parameter of 0.5, 
generator noise sampled from a 100-dimensional unit normal distribution, a mini-batch size of 64, and 600 epochs. 
The Paper implementation is based on the dcgan.torch2 framework.

\subsubsection*{Vanilla GAN: Generative Adversarial Network without Conditional Latent Space}
The Vanilla GAN consists of a Generator that only relies on the noise vector $z$ and a Discriminator 
that evaluates whether the input image is real or generated, without any dependence on a conditional latent space.

\subsubsection*{GAN-CLS: Generative Adversarial Network with Conditional Latent Space}
GAN-CLS utilizes a text encoder, $\varphi$ (such as CLIP or GoogLeNet ), to create embeddings from text descriptions ${t}$ 
that are related to images ${x}$, resulting in embeddings ${h}$ and $\hat{h}$ 
for mismatched descriptions. 
A random noise vector ${z}$ is sampled from a Gaussian distribution and passed 
through the Generator to generate an image $\hat{x}$. 
The Discriminator computes a score ${s_r}$ for the real image and its corresponding text embedding. 
For a fake image, the Discriminator calculates ${s_f}$.\\ 
The Discriminator loss $L_D$ is computed as:
\[
L_D = d\_loss\_f + d\_loss\_w + d\_loss\_r
\]
The Generator's loss, $L_G$, consists of three parts: BCELoss, MSELoss, and L1Loss, and is calculated as:
\[
L_G = BCELoss(s_f, \text{real\_label}) 
\]
\[
  + MSELoss(q_r, q_f) + L1Loss(\hat{x}, x)
\]
The losses are backpropagated and the models are updated with the learning rate $\alpha$.

\subsubsection*{GAN-INT: Generative Adversarial Network Interpolated}
This architecture improves upon GAN-CLS by interpolating between two text embeddings. 
The Generator is conditioned on the interpolated embeddings to generate more diverse images. 
Although this modification has not been implemented yet, it can be added with minor adjustments.

\[
\mathbb{E}_{h1,h2 \sim p_{\text{data}} } [\log( 1 - D(G( z , \beta h1 + (1-\beta) h2 )))]
\]

\subsubsection*{WGAN: Wasserstein GAN - Our Architecture}
WGAN introduces the Wasserstein distance to improve training stability. 
The Discriminator computes the difference between real and fake scores, 
and the Discriminator's weights are clipped to satisfy the Lipschitz constraint. 
The Generator is trained to maximize the Discriminator's score. 
The WGAN training steps are as follows:
\begin{algorithm}
    \caption{WGAN training algorithm with step size $\alpha$}
    \begin{algorithmic}
        \Require minibatch images $x$, matching text $t$, mis-matching text $\tilde{t}$, training batches $S$, Discriminator iterations $NI$
        \For{$n = 1$ to $S$}
            \State $h \leftarrow \varphi(t)$ \hfill \{Encode matching description\}
            \For{$i = 1$ to $NI$}
                \State $z \sim \mathcal{N}(0, 1)^Z$
                \State $\hat{x} \leftarrow G(z, h)$
                \State $s_r \leftarrow D(x, h)$
                \State $s_f \leftarrow D(\hat{x}, h)$
                \State $L_D \leftarrow s_r - s_f$
                \State Clip weights of $D$ within $[-c, c]$
            \EndFor
            \State $L_G \leftarrow -s_f$
        \EndFor
    \end{algorithmic}
\end{algorithm}
\newpage
The Wasserstein loss encourages the Generator to produce images 
that the Discriminator cannot easily distinguish from real ones. 
The iterative process ensures stable GAN training and minimizes 
the Wasserstein distance between real and generated data distributions.


\begin{comment}

\subsubsection*{Generator Part}
The Generator creates 64x64 RGB images from text descriptions and noise vectors. 
It consists of three main steps:

    \subparagraph{Projection}: 
    The text embedding $\varphi(t)$ (encoded with CLIP) is reduced to a 128-dimensional vector $p$ using:
    \begin{itemize}
        \item A fully-connected layer,
        \item BatchNorm,
        \item Leaky-ReLU as activation function.
    \end{itemize}

    \subparagraph{Concatenation}: 
    The noise vector $z$ (dimension 100) is concatenated with $p$, forming a latent vector $h$ (dimension 228).
    
    \subparagraph{Final Network}: The latent vector $h$ is processed through:
    \begin{itemize}
        \item Transpose convolutional layers to upscale dimensions,
        \item Tanh activation function to normalize RGB values to $[-1, 1]$,
    \end{itemize}
    producing a 64x64 image.

\subsubsection*{Discriminator Part}
The Discriminator evaluates if a generated image $\hat{x}$ matches its corresponding text description $\varphi(t)$. It has three main steps:

\begin{itemize}
    \item \textbf{Down-Sampling}: The image is reduced using convolutional layers, producing a feature map $q$.
    
    \item \textbf{Text Projection \& Concatenation}:
    \begin{itemize}
        \item The text embedding $\varphi(t)$ is projected to a 128-dimensional vector $p$ via a fully-connected layer, reshaped to match $q$.
        \item The reshaped ("squeezed") vector $\hat{p}$ is concatenated with $q$, forming a latent vector $c$.
    \end{itemize}

    \item \textbf{Final Network}: The latent vector $c$ is passed through:
    \begin{itemize}
        \item A convolutional layer,
        \item Sigmoid activation function,
    \end{itemize}
    to output a 121-dimensional vector representing discriminator scores.
\end{itemize}



\paragraph{Shared Characteristics of the two Encoders and difference } 
Both encoders map their inputs to a 1,024-dimensional vector space and
the text encoder uses a combination of a convolutional neural network (CNN) and a recurrent neural network (RNN) 
to capture semantic meaning from text.
Furthermore , pretraining on large datasets speeds up convergence and improves generalization, 
even to unseen datasets such as MS COCO.
This approach highlights the value of using modern models like CLIP to replace traditional encoders, 
achieving better generalization and performance across diverse image-text datasets.


\end{comment}